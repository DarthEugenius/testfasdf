\documentclass[a4paper,12pt]{extarticle}
% !TEX encoding = IBM866
\usepackage[utf8]{inputenc}
\usepackage[T2A]{fontenc}
\usepackage{listings}
\usepackage[cp866]{inputenc}
\usepackage[dvipsnames]{xcolor}
\usepackage{algorithm2e}
\usepackage{array}
\usepackage[final]{graphicx}
\usepackage{float}
\usepackage{graphicx}
\usepackage{moreverb}
\usepackage{multirow}
\usepackage[cp866]{inputenc}
\usepackage[russian]{babel}
\usepackage{amsthm,amsmath,amsfonts,amssymb}
\usepackage[final]{graphicx}
\usepackage{float}
\usepackage{geometry}
\geometry
{
a4paper,
total={210mm,297mm},
left=20mm,
right=20mm,
top=25mm,
bottom=20mm,
}

\usepackage{tikz}
\usepackage{pgfplots}
\usepackage[final]{showkeys}

\usepackage{caption}
\DeclareCaptionLabelSeparator{dot}{. }
\captionsetup{labelsep=dot}

\usepackage{float}
\usepackage{hyperref}
\usepackage{lastpage}
\usepackage{fancyhdr}
\usepackage{python}
\usepackage{listings}
\lstloadlanguages{Python}
\usepackage{bm}

% \begin{document}
\begin{document}


    \begin{titlepage}

        \begin{center}
            \centerline{\Large\rm Министерство науки и высшего образования}
            \centerline{\Large\rm Федеральное государственное бюджетное образовательное}
            \centerline{\Large\rm учреждение высшего образования}
            \centerline{\Large\rm <<Московский государственный технический университет}
            \centerline{\Large\rm имени~Н.~Э.~Баумана}
            \centerline{\Large\rm (национальный исследовательский университет)>>}
            \centerline{\Large\rm (МГТУ~им.~Н.~Э.~Баумана)}
            \hrulefill
        \end{center}

        \begin{figure}[h!]
            \centering
            %			\includegraphics[height=0.4\linewidth]{picture.0}
        \end{figure}

        \begin{center}
            \centerline{\Large\rm Факультет <<Фундаментальные науки>>}
            \centerline{\Large\rm Кафедра ФН1 <<Высшая математика>>}
            \centerline{\Large\rm Дисциплина <<Методы вычислений>>}
        \end{center}

        \begin{center}
            \textsc{\textbf{\Huge Отчет}}\\
            \textsc{\textbf{\large по лабораторной работе №4}}\\
            \textsc{\textbf{\large аппроксимация функции}}\\
        \end{center}

        \vspace{3em}

        {
        \large
        \hbox to 17cm {Преподаватель \hspace{45pt} \hrulefill \hspace{60pt} Коновалов~Я.~Ю.}
        \vspace{-7pt}
        \hbox{{\small\it \hspace{178pt} подпись, инициалы}}
        \hbox{}
        \hbox to 17cm {Студенты группы ФН1--51Б \hrulefill \hspace{1pt} Терновой~Е.~А. и Гончаров~М.~В.}
        \vspace{-7pt}
        \hbox{{\small\it \hspace{178pt} подпись, инициалы}}
        }


        \vspace{\fill}

        \begin{center}
            \large	Москва \\2020
        \end{center}

    \end{titlepage}

    \setcounter{page}{2}
    \tableofcontents
    \vspace{\baselineskip}

    \newpage
    \section{Задание}

    Аппроксимировать с помощью метода Лагранжа,
    интерполяции и метода наименьших квадратов функции:\\

    \begin{table}[ht]
        \centering
        \begin{tabular}{p{10cm}cp{3cm}}
            $f(x)=\sqrt{9x-2}+\sin(3x + \frac{\pi}{3})$& \hfill & $  x \in \left[1, 6\right] $\\
        \end{tabular}
    \end{table}


    Взять $5,10,20$ разбиений для каждого метода, затем в серединах получившихся отрезков вычислить значение функции и функции-апроксимации, найти их модуль разности и найти среднеквадратичную погрешность и максимум из этих погрешностей
    \section{Метод интерполяции Лагранжа}
    \listinginput[1]{1}{LagrangeInterpolation.java}
    \newpage \subsection{Результаты работы программы}
    При {n = 5}
    \begin{table}[H]
        \centering
        \begin{tabular}{|l|l|l|l|}
            \hline
            x      & \begin{tabular}[c]{@{}l@{}}Значение\\ исходной\\ функции \end{tabular}& \begin{tabular}[c]{@{}l@{}}Значение \\ аппроксимирующей \\функции\end{tabular} & Модуль разности    \\
            \hline1.625 & 3.9063660998495324 & 2.1180951978227323 & 1.7882709020268002\\ \hline
            2.875 & 5.1311080188931095 & 5.047569903191512 & 0.08353811570159753\\ \hline
            4.125 & 5.171521486230043 & 6.517837421228592 & 1.3463159349985485\\ \hline
            5.375 & 7.804246872120658 & 5.609347791788444 & 2.194899080332214\\ \hline
        \end{tabular}
        \centering
        \begin{tabular}{|l|l|}
            \hline
            \begin{tabular}[c]{@{}l@{}}Среднее значение\\модуля разности\end{tabular} & \begin{tabular}[c]{@{}l@{}}Среднеквадратическое\\отклонение\end{tabular}  \\
            \hline
            1.3532560082647902 & 2.4587600013712336\\
            \hline
        \end{tabular}\end{table}
    \begin{figure}[h!]
        \centering
        \includegraphics[height=0.7\linewidth]{LagrangeInterpolation5Nodes.png}
    \end{figure}


    \newpage При {n = 10}
    \begin{table}[H]
        \centering
        \begin{tabular}{|l|l|l|l|}
            \hline
            x      & \begin{tabular}[c]{@{}l@{}}Значение\\ исходной\\ функции \end{tabular}& \begin{tabular}[c]{@{}l@{}}Значение \\ аппроксимирующей \\функции\end{tabular} & Модуль разности    \\
            \hline1.2777777777777777 & 4.06810442385949 & 3.5334426805521337 & 0.5346617433073559\\ \hline
            1.8333333333333335 & 3.5469306882867353 & 3.6488446051030947 & 0.10191391681635942\\ \hline
            2.3888888888888893 & 3.479942253266922 & 3.4447644719191826 & 0.03517778134773941\\ \hline
            2.9444444444444446 & 5.389885979554769 & 5.408516483418956 & 0.018630503864186743\\ \hline
            3.5 & 6.283065185668083 & 6.268933255045663 & 0.014131930622419908\\ \hline
            4.055555555555555 & 5.270480856949955 & 5.285478047515501 & 0.014997190565546603\\ \hline
            4.611111111111111 & 5.5487064266280735 & 5.526194412583605 & 0.022512014044468565\\ \hline
            5.166666666666666 & 7.414963846316273 & 7.465413229017852 & 0.050449382701578394\\ \hline
            5.722222222222221 & 7.629357883193977 & 7.434321207494628 & 0.1950366756993489\\ \hline
        \end{tabular}
        \centering
        \begin{tabular}{|l|l|}
            \hline
            \begin{tabular}[c]{@{}l@{}}Среднее значение\\модуля разности\end{tabular} & \begin{tabular}[c]{@{}l@{}}Среднеквадратическое\\отклонение\end{tabular}  \\
            \hline
            0.10972345988544488 & 0.03770556235769262\\
            \hline
        \end{tabular}\end{table}
    \begin{figure}[h!]
        \centering
        \includegraphics[height=0.7\linewidth]{LagrangeInterpolation10Nodes.png}
    \end{figure}


    ~\newline
    ~\newline
    \newpageПри {n = 20}

    \begin{table}[H]
        \centering
        \begin{tabular}{|l|l|l|l|}
            \hline
            x      & \begin{tabular}[c]{@{}l@{}}Значение\\ исходной\\ функции \end{tabular}& \begin{tabular}[c]{@{}l@{}}Значение \\ аппроксимирующей \\функции\end{tabular} & Модуль разности    \\
            \hline1.131578947368421 & 3.8244554666413197 & 3.824471717362743 & 1.6250721423194392E-5\\ \hline
            1.394736842105263 & 4.116787579856203 & 4.116786249836062 & 1.3300201402444145E-6\\ \hline
            1.657894736842105 & 3.8538925642006365 & 3.8538927550252797 & 1.9082464319808423E-7\\ \hline
            1.9210526315789471 & 3.4070859630079693 & 3.4070859225244994 & 4.048346990970231E-8\\ \hline
            2.1842105263157894 & 3.2342530606473807 & 3.2342530723575535 & 1.1710172831413956E-8\\ \hline
            2.447368421052631 & 3.6149723485483243 & 3.614972344138934 & 4.409390186310702E-9\\ \hline
            2.7105263157894735 & 4.488780012421658 & 4.488780014521663 & 2.1000055028252973E-9\\ \hline
            2.973684210526315 & 5.493370469332582 & 5.4933704680906885 & 1.2418936989888607E-9\\ \hline
            3.2368421052631575 & 6.180644915873133 & 6.180644916774614 & 9.01480667891974E-10\\ \hline
            3.499999999999999 & 6.283065185668084 & 6.283065184869942 & 7.981419969382841E-10\\ \hline
            3.7631578947368416 & 5.872897032504954 & 5.872897033365079 & 8.601244161354771E-10\\ \hline
            4.026315789473683 & 5.320230729188642 & 5.3202307280581325 & 1.130509907909527E-9\\ \hline
            4.289473684210526 & 5.074669969235286 & 5.074669971059055 & 1.8237686916222628E-9\\ \hline
            4.552631578947368 & 5.399871981641104 & 5.399871977988123 & 3.6529810287788678E-9\\ \hline
            4.815789473684211 & 6.2180013255054005 & 6.218001334759265 & 9.253864341474127E-9\\ \hline
            5.078947368421053 & 7.156131446655761 & 7.156131416139653 & 3.051610786286574E-8\\ \hline
            5.342105263157896 & 7.767158390901764 & 7.767158528134152 & 1.3723238811991223E-7\\ \hline
            5.605263157894738 & 7.7945528353675915 & 7.794551922305299 & 9.130622924402587E-7\\ \hline
            5.8684210526315805 & 7.324341056153398 & 7.324351721462135 & 1.0665308737500823E-5\\ \hline
        \end{tabular}
        \centering
        \begin{tabular}{|l|l|}
            \hline
            \begin{tabular}[c]{@{}l@{}}Среднее значение\\модуля разности\end{tabular} & \begin{tabular}[c]{@{}l@{}}Среднеквадратическое\\отклонение\end{tabular}  \\
            \hline
            1.5576869229337233E-6 & 2.002607783031335E-11\\
            \hline
        \end{tabular}\end{table}
    \begin{figure}[h!]
        \centering
        \includegraphics[height=0.5\linewidth]{LagrangeInterpolation20Nodes.png}
    \end{figure}

    \newpage \section{Метод сплайнов}
    \listinginput[1]{1}{SplineInterpolationMethod.java}
    \newpage \subsection{Результаты работы программы}

    При {n = 5}
    \begin{table}[H]
        \centering
        \begin{tabular}{|l|l|l|l|}
            \hline
            x      & \begin{tabular}[c]{@{}l@{}}Значение\\ исходной\\ функции \end{tabular}& \begin{tabular}[c]{@{}l@{}}Значение \\ аппроксимирующей \\функции\end{tabular} & Модуль разности    \\
            \hline1.625 & 3.9063660998495324 & 2.937313096357878 & 0.9690530034916542\\ \hline
            2.875 & 5.1311080188931095 & 4.837500115311667 & 0.2936079035814423\\ \hline
            4.125 & 5.171521486230043 & 6.365239505857842 & 1.1937180196277986\\ \hline
            5.375 & 7.804246872120658 & 6.256150072796305 & 1.5480967993243535\\ \hline
        \end{tabular}
        \centering
        \begin{tabular}{|l|l|}
            \hline
            \begin{tabular}[c]{@{}l@{}}Среднее значение\\модуля разности\end{tabular} & \begin{tabular}[c]{@{}l@{}}Среднеквадратическое\\отклонение\end{tabular}  \\
            \hline
            1.001118931506312 & 1.2117089337710265\\
            \hline
        \end{tabular}\end{table}
    \begin{figure}[h!]
        \centering
        \includegraphics[height=0.7\linewidth]{SplineInterpolationMethod5Nodes.png}
    \end{figure}

\newpage    При {n = 10}

    \begin{table}[H]
        \centering
        \begin{tabular}{|l|l|l|l|}
            \hline
            x      & \begin{tabular}[c]{@{}l@{}}Значение\\ исходной\\ функции \end{tabular}& \begin{tabular}[c]{@{}l@{}}Значение \\ аппроксимирующей \\функции\end{tabular} & Модуль разности    \\
            \hline1.2777777777777777 & 4.06810442385949 & 3.893577797583818 & 0.1745266262756715\\ \hline
            1.8333333333333335 & 3.5469306882867353 & 3.5937188916441114 & 0.0467882033573761\\ \hline
            2.3888888888888893 & 3.479942253266922 & 3.5051020864193907 & 0.025159833152468725\\ \hline
            2.9444444444444446 & 5.389885979554769 & 5.376050046656346 & 0.013835932898423664\\ \hline
            3.5 & 6.283065185668083 & 6.250661209880627 & 0.032403975787456574\\ \hline
            4.055555555555555 & 5.270480856949955 & 5.292658879131686 & 0.02217802218173137\\ \hline
            4.611111111111111 & 5.5487064266280735 & 5.578370559812557 & 0.029664133184483354\\ \hline
            5.166666666666666 & 7.414963846316273 & 7.379037092205483 & 0.035926754110790604\\ \hline
            5.722222222222221 & 7.629357883193977 & 7.63740080048011 & 0.008042917286132933\\ \hline
        \end{tabular}
        \centering
        \begin{tabular}{|l|l|}
            \hline
            \begin{tabular}[c]{@{}l@{}}Среднее значение\\модуля разности\end{tabular} & \begin{tabular}[c]{@{}l@{}}Среднеквадратическое\\отклонение\end{tabular}  \\
            \hline
            0.0431695998038372 & 0.004138932531970204\\
            \hline
        \end{tabular}\end{table}

    \begin{figure}[h!]
        \centering
        \includegraphics[height=0.7\linewidth]{SplineInterpolationMethod10Nodes.png}
    \end{figure}
    ~\newline
    ~\newline

    \newpage При {n = 20}

    \begin{table}[H]
        \centering
        \begin{tabular}{|l|l|l|l|}
            \hline
            x      & \begin{tabular}[c]{@{}l@{}}Значение\\ исходной\\ функции \end{tabular}& \begin{tabular}[c]{@{}l@{}}Значение \\ аппроксимирующей \\функции\end{tabular} & Модуль разности    \\
            \hline1.131578947368421 & 3.8244554666413197 & 3.796383292825589 & 0.028072173815730572\\ \hline
            1.394736842105263 & 4.116787579856203 & 4.123009281979204 & 0.006221702123001549\\ \hline
            1.657894736842105 & 3.8538925642006365 & 3.8516593235214125 & 0.0022332406792240356\\ \hline
            1.9210526315789471 & 3.4070859630079693 & 3.4082023049862955 & 0.0011163419783262185\\ \hline
            2.1842105263157894 & 3.2342530606473807 & 3.2352568456610427 & 0.0010037850136619397\\ \hline
            2.447368421052631 & 3.6149723485483243 & 3.616024451457656 & 0.0010521029093317757\\ \hline
            2.7105263157894735 & 4.488780012421658 & 4.489058181972087 & 2.7816955042947455E-4\\ \hline
            2.973684210526315 & 5.493370469332582 & 5.492765194511405 & 6.052748211775594E-4\\ \hline
            3.2368421052631575 & 6.180644915873133 & 6.179500521931908 & 0.001144393941224564\\ \hline
            3.499999999999999 & 6.283065185668084 & 6.282062978356134 & 0.0010022073119504427\\ \hline
            3.7631578947368416 & 5.872897032504954 & 5.872629432772201 & 2.675997327532542E-4\\ \hline
            4.026315789473683 & 5.320230729188642 & 5.320855773183306 & 6.250439946633435E-4\\ \hline
            4.289473684210526 & 5.074669969235286 & 5.075821114582095 & 0.001151145346809379\\ \hline
            4.552631578947368 & 5.399871981641104 & 5.400856982814935 & 9.850011738317122E-4\\ \hline
            4.815789473684211 & 6.2180013255054005 & 6.218280387221027 & 2.7906171562630533E-4\\ \hline
            5.078947368421053 & 7.156131446655761 & 7.155379989998384 & 7.514566573769699E-4\\ \hline
            5.342105263157896 & 7.767158390901764 & 7.766416691721737 & 7.416991800264228E-4\\ \hline
            5.605263157894738 & 7.7945528353675915 & 7.792036710623157 & 0.0025161247444343005\\ \hline
            5.8684210526315805 & 7.324341056153398 & 7.32983542024025 & 0.005494364086851888\\ \hline
        \end{tabular}
        \centering
        \begin{tabular}{|l|l|}
            \hline
            \begin{tabular}[c]{@{}l@{}}Среднее значение\\модуля разности\end{tabular} & \begin{tabular}[c]{@{}l@{}}Среднеквадратическое\\отклонение\end{tabular}  \\
            \hline
            0.0029232046724437743 & 4.622798180944326E-5\\
            \hline
        \end{tabular}\end{table}
    \begin{figure}[h!]
        \centering
        \includegraphics[height=0.5\linewidth]{SplineInterpolationMethod20Nodes.png}
    \end{figure}

    \newpage\section{Метод наименьших квадратов}
    \listinginput[1]{1}{LeastSquaresMethod.java}
   \newpage \subsection{Результаты работы программы}
% \centering
    При {n = 5}\\
    \begin{table}[H]
        \centering
        \begin{tabular}{|l|l|l|l|}
            \hline
            x      & \begin{tabular}[c]{@{}l@{}}Значение\\ исходной\\ функции \end{tabular}& \begin{tabular}[c]{@{}l@{}}Значение \\ аппроксимирующей \\функции\end{tabular} & Модуль разности    \\
            \hline1.625 & 3.9063660998495324 & 3.4354240320584783 & 0.4709420677910541\\ \hline
            2.875 & 5.1311080188931095 & 4.578690148633156 & 0.5524178702599531\\ \hline
            4.125 & 5.171521486230043 & 6.048957666670059 & 0.8774361804400161\\ \hline
            5.375 & 7.804246872120658 & 6.926676626023994 & 0.8775702460966643\\ \hline
        \end{tabular}
        \centering
        \begin{tabular}{|l|l|}
            \hline
            \begin{tabular}[c]{@{}l@{}}Среднее значение\\модуля разности\end{tabular} & \begin{tabular}[c]{@{}l@{}}Среднеквадратическое\\отклонение\end{tabular}  \\
            \hline
            0.6945915911469219 & 0.5167439305442952\\
            \hline
        \end{tabular}\end{table}
    \begin{figure}[h!]
        \centering
        \includegraphics[height=0.7\linewidth]{LeastSquaresMethod5Nodes.png}
    \end{figure}
    \newpageПри {n = 10}\\

    \begin{table}[H]
        \centering
        \begin{tabular}{|l|l|l|l|}
            \hline
            x      & \begin{tabular}[c]{@{}l@{}}Значение\\ исходной\\ функции \end{tabular}& \begin{tabular}[c]{@{}l@{}}Значение \\ аппроксимирующей \\функции\end{tabular} & Модуль разности    \\
            \hline1.2777777777777777 & 4.06810442385949 & 3.518258276440912 & 0.5498461474185778\\ \hline
            1.8333333333333335 & 3.5469306882867353 & 3.8138831598891496 & 0.2669524716024143\\ \hline
            2.3888888888888893 & 3.479942253266922 & 4.243072768010975 & 0.7631305147440535\\ \hline
            2.9444444444444446 & 5.389885979554769 & 4.762447487972166 & 0.6274384915826037\\ \hline
            3.5 & 6.283065185668083 & 5.328627706938501 & 0.9544374787295826\\ \hline
            4.055555555555555 & 5.270480856949955 & 5.898233812075761 & 0.6277529551258061\\ \hline
            4.611111111111111 & 5.5487064266280735 & 6.427886190549723 & 0.8791797639216492\\ \hline
            5.166666666666666 & 7.414963846316273 & 6.874205229526166 & 0.5407586167901073\\ \hline
            5.722222222222221 & 7.629357883193977 & 7.193811316170867 & 0.4355465670231098\\ \hline
        \end{tabular}
        \centering
        \begin{tabular}{|l|l|}
            \hline
            \begin{tabular}[c]{@{}l@{}}Среднее значение\\модуля разности\end{tabular} & \begin{tabular}[c]{@{}l@{}}Среднеквадратическое\\отклонение\end{tabular}  \\
            \hline
            0.6272270007708782 & 0.4344160084021362\\
            \hline
        \end{tabular}\end{table}
    \begin{figure}[h!]
        \centering
        \includegraphics[height=0.7\linewidth]{LeastSquaresMethod10Nodes.png}
    \end{figure}
   \newpage При {n = 20}\\

    \begin{table}[H]
        \centering
        \begin{tabular}{|l|l|l|l|}
            \hline
            x      & \begin{tabular}[c]{@{}l@{}}Значение\\ исходной\\ функции \end{tabular}& \begin{tabular}[c]{@{}l@{}}Значение \\ аппроксимирующей \\функции\end{tabular} & Модуль разности    \\
            \hline1.131578947368421 & 3.8244554666413197 & 3.5558947030700034 & 0.2685607635713163\\ \hline
            1.394736842105263 & 4.116787579856203 & 3.6434958152021863 & 0.47329176465401623\\ \hline
            1.657894736842105 & 3.8538925642006365 & 3.7662262966840965 & 0.08766626751654005\\ \hline
            1.9210526315789471 & 3.4070859630079693 & 3.9203059582661774 & 0.5132199952582082\\ \hline
            2.1842105263157894 & 3.2342530606473807 & 4.10195461069887 & 0.8677015500514895\\ \hline
            2.447368421052631 & 3.6149723485483243 & 4.307392064732617 & 0.6924197161842924\\ \hline
            2.7105263157894735 & 4.488780012421658 & 4.5328381311178605 & 0.0440581186962028\\ \hline
            2.973684210526315 & 5.493370469332582 & 4.774512620605039 & 0.7188578487275432\\ \hline
            3.2368421052631575 & 6.180644915873133 & 5.028635343944599 & 1.1520095719285335\\ \hline
            3.499999999999999 & 6.283065185668084 & 5.291426111886979 & 0.9916390737811049\\ \hline
            3.7631578947368416 & 5.872897032504954 & 5.559104735182625 & 0.3137922973223297\\ \hline
            4.026315789473683 & 5.320230729188642 & 5.827891024581973 & 0.5076602953933307\\ \hline
            4.289473684210526 & 5.074669969235286 & 6.09400479083547 & 1.019334821600184\\ \hline
            4.552631578947368 & 5.399871981641104 & 6.353665844693553 & 0.9537938630524492\\ \hline
            4.815789473684211 & 6.2180013255054005 & 6.60309399690667 & 0.3850926714012699\\ \hline
            5.078947368421053 & 7.156131446655761 & 6.838509058225258 & 0.31762238843050294\\ \hline
            5.342105263157896 & 7.767158390901764 & 7.056130839399761 & 0.7110275515020028\\ \hline
            5.605263157894738 & 7.7945528353675915 & 7.252179151180618 & 0.5423736841869733\\ \hline
            5.8684210526315805 & 7.324341056153398 & 7.422873804318274 & 0.09853274816487634\\ \hline
        \end{tabular}
        \centering
        \begin{tabular}{|l|l|}
            \hline
            \begin{tabular}[c]{@{}l@{}}Среднее значение\\модуля разности\end{tabular} & \begin{tabular}[c]{@{}l@{}}Среднеквадратическое\\отклонение\end{tabular}  \\
            \hline
            0.5609818416538509 & 0.4206474855554798\\
            \hline
        \end{tabular}\end{table}
    \begin{figure}[h!]
        \centering
        \includegraphics[height=0.5\linewidth] {LeastSquaresMethod20Nodes.png}
    \end{figure}
% \hline
\end{document}
