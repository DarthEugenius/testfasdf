\documentclass[a4paper,12pt]{extarticle}
% !TEX encoding = IBM866
\usepackage[utf8]{inputenc}
\usepackage[T2A]{fontenc}
\usepackage{listings}
\usepackage[cp866]{inputenc}
\usepackage[dvipsnames]{xcolor}
\usepackage{algorithm2e}
\usepackage{array}
\usepackage[final]{graphicx}
\usepackage{float}
\usepackage{graphicx}
\usepackage{moreverb}
\usepackage{multirow}
\usepackage[cp866]{inputenc}
\usepackage[russian]{babel}
\usepackage{amsthm,amsmath,amsfonts,amssymb}
\usepackage[final]{graphicx}
\usepackage{float}
\usepackage{geometry}
\geometry
{
a4paper,
total={210mm,297mm},
left=20mm,
right=20mm,
top=25mm,
bottom=20mm,
}

\usepackage{tikz}
\usepackage{pgfplots}
\usepackage[final]{showkeys}

\usepackage{caption}
\DeclareCaptionLabelSeparator{dot}{. }
\captionsetup{labelsep=dot}

\usepackage{float}
\usepackage{hyperref}
\usepackage{lastpage}
\usepackage{fancyhdr}
\usepackage{python}
\usepackage{listings}
\lstloadlanguages{Python}
\usepackage{bm}

% \begin{document}
\begin{document}


    \begin{titlepage}

        \begin{center}
            \centerline{\Large\rm Министерство науки и высшего образования}
            \centerline{\Large\rm Федеральное государственное бюджетное образовательное}
            \centerline{\Large\rm учреждение высшего образования}
            \centerline{\Large\rm <<Московский государственный технический университет}
            \centerline{\Large\rm имени~Н.~Э.~Баумана}
            \centerline{\Large\rm (национальный исследовательский университет)>>}
            \centerline{\Large\rm (МГТУ~им.~Н.~Э.~Баумана)}
            \hrulefill
        \end{center}

        \begin{figure}[h!]
            \centering
            %			\includegraphics[height=0.4\linewidth]{picture.0}
        \end{figure}

        \begin{center}
            \centerline{\Large\rm Факультет <<Фундаментальные науки>>}
            \centerline{\Large\rm Кафедра ФН1 <<Высшая математика>>}
            \centerline{\Large\rm Дисциплина <<Методы вычислений>>}
        \end{center}

        \begin{center}
            \textsc{\textbf{\Huge Отчет}}\\
            \textsc{\textbf{\large по лабораторной работе №4}}\\
            \textsc{\textbf{\large аппроксимация функции}}\\
        \end{center}

        \vspace{3em}

        {
        \large
        \hbox to 17cm {Преподаватель \hspace{45pt} \hrulefill \hspace{60pt} Коновалов~Я.~Ю.}
        \vspace{-7pt}
        \hbox{{\small\it \hspace{178pt} подпись, инициалы}}
        \hbox{}
        \hbox to 17cm {Студенты группы ФН1--51Б \hrulefill \hspace{1pt} Терновой~Е.~А. и Гончаров~М.~В.}
        \vspace{-7pt}
        \hbox{{\small\it \hspace{178pt} подпись, инициалы}}
        }


        \vspace{\fill}

        \begin{center}
            \large	Москва \\2020
        \end{center}

    \end{titlepage}

    \setcounter{page}{2}
    \tableofcontents
    \vspace{\baselineskip}

    \newpage
    \section{Задание}

    Аппроксимировать с помощью метода Лагранжа,
    интерполяции и метода наименьших квадратов функции:\\

    \begin{table}[ht]
        \centering
        \begin{tabular}{p{10cm}cp{3cm}}
            $f(x)=\sqrt{9x-2}+\sin(3x + \frac{\pi}{3})$& \hfill & $  x \in \left[1, 6\right] $\\
        \end{tabular}
    \end{table}


    Взять $5,10,20$ разбиений для каждого метода, затем в серединах получившихся отрезков вычислить значение функции и функции-апроксимации, найти их модуль разности и найти среднеквадратичную погрешность и максимум из этих погрешностей
    \section{Метод интерполяции Лагранжа}
    \listinginput[1]{1}{LagrangeInterpolation.java}
    \newpage \subsection{Результаты работы программы}
    При {n = 5}
    \begin{table}[H]
        \centering
        \begin{tabular}{|l|l|l|l|}
            \hline
            x      & \begin{tabular}[c]{@{}l@{}}Значение\\ исходной\\ функции \end{tabular}& \begin{tabular}[c]{@{}l@{}}Значение \\ аппроксимирующей \\функции\end{tabular} & Модуль разности    \\
            \hline1.625 & 3.199969101926415 & 4.96266672092871 & 1.762697619002295\\ \hline
            2.875 & 4.641302123952923 & 4.731779037635981 & 0.09047691368305788\\ \hline
            4.125 & 6.681748104899654 & 5.330306108044837 & 1.3514419968548168\\ \hline
            5.375 & 5.815591599755577 & 8.020007542678016 & 2.204415942922439\\ \hline
        \end{tabular}
        \centering
        \begin{tabular}{|l|l|}
            \hline
            \begin{tabular}[c]{@{}l@{}}Среднее значение\\модуля разности\end{tabular} & \begin{tabular}[c]{@{}l@{}}Среднеквадратическое\\отклонение\end{tabular}  \\
            \hline
            1.352258118115652 & 2.4502835220548826\\
            \hline
        \end{tabular}\end{table}
    \begin{figure}[h!]
        \centering
        \includegraphics[height=0.7\linewidth]{LagrangeInterpolation5Nodes.png}
    \end{figure}


    \newpage При {n = 10}
    \begin{table}[H]
        \centering
        \begin{tabular}{|l|l|l|l|}
            \hline
            x      & \begin{tabular}[c]{@{}l@{}}Значение\\ исходной\\ функции \end{tabular}& \begin{tabular}[c]{@{}l@{}}Значение \\ аппроксимирующей \\функции\end{tabular} & Модуль разности    \\
            \hline1.2777777777777777 & 2.0963095791094863 & 2.630437900558312 & 0.5341283214488257\\ \hline
            1.8333333333333335 & 4.068842417577174 & 3.966995892414827 & 0.1018465251623466\\ \hline
            2.3888888888888893 & 5.351818613060926 & 5.386978845080125 & 0.03516023201919882\\ \hline
            2.9444444444444446 & 4.509608957056896 & 4.490986216950864 & 0.018622740106032154\\ \hline
            3.5 & 4.579715305532132 & 4.593841842045163 & 0.01412653651303053\\ \hline
            4.055555555555555 & 6.4768592675207755 & 6.461867807905699 & 0.014991459615076863\\ \hline
            4.611111111111111 & 7.021098663348462 & 7.043601262494464 & 0.02250259914600239\\ \hline
            5.166666666666666 & 5.926700217810059 & 5.876276161251828 & 0.050424056558231456\\ \hline
            5.722222222222221 & 6.441889396276311 & 6.636795984579647 & 0.1949065883033363\\ \hline
        \end{tabular}
        \centering
        \begin{tabular}{|l|l|}
            \hline
            \begin{tabular}[c]{@{}l@{}}Среднее значение\\модуля разности\end{tabular} & \begin{tabular}[c]{@{}l@{}}Среднеквадратическое\\отклонение\end{tabular}  \\
            \hline
            0.10963433987467565 & 0.03763451781483216\\
            \hline
        \end{tabular}\end{table}
    \begin{figure}[h!]
        \centering
        \includegraphics[height=0.7\linewidth]{LagrangeInterpolation10Nodes.png}
    \end{figure}


    ~\newline
    ~\newline
    \newpageПри {n = 20}

    \begin{table}[H]
        \centering
        \begin{tabular}{|l|l|l|l|}
            \hline
            x      & \begin{tabular}[c]{@{}l@{}}Значение\\ исходной\\ функции \end{tabular}& \begin{tabular}[c]{@{}l@{}}Значение \\ аппроксимирующей \\функции\end{tabular} & Модуль разности    \\
            \hline1.131578947368421 & 1.8971563783952827 & 1.8971390113833506 & 1.736701193211232E-5\\ \hline
            1.394736842105263 & 2.380175273338317 & 2.3801766761517507 & 1.4028134338595066E-6\\ \hline
            1.657894736842105 & 3.3352805820183207 & 3.335280382490388 & 1.995279328426136E-7\\ \hline
            1.9210526315789471 & 4.413265354985494 & 4.4132653970568825 & 4.207138815104372E-8\\ \hline
            2.1842105263157894 & 5.1700065109846385 & 5.1700064988699275 & 1.2114711012145563E-8\\ \hline
            2.447368421052631 & 5.335182016483303 & 5.335182021029409 & 4.546105714098303E-9\\ \hline
            2.7105263157894735 & 4.975835530135844 & 4.975835527976475 & 2.1593686838627946E-9\\ \hline
            2.973684210526315 & 4.459148386662987 & 4.459148387937307 & 1.2743202049136926E-9\\ \hline
            3.2368421052631575 & 4.236951533859102 & 4.236951532935612 & 9.234897291321431E-10\\ \hline
            3.499999999999999 & 4.57971530553213 & 4.579715306348676 & 8.165468301513101E-10\\ \hline
            3.7631578947368416 & 5.417527415252814 & 5.41752741437378 & 8.790337346908927E-10\\ \hline
            4.026315789473683 & 6.38222065592946 & 6.38222065708387 & 1.1544099010052378E-9\\ \hline
            4.289473684210526 & 7.025786742458655 & 7.025786740597508 & 1.861146792236923E-9\\ \hline
            4.552631578947368 & 7.085909405284797 & 7.085909409010901 & 3.726103869894359E-9\\ \hline
            4.815789473684211 & 6.6415640268940175 & 6.641564017457958 & 9.436059045242473E-9\\ \hline
            5.078947368421053 & 6.066655901825569 & 6.0666559329365946 & 3.1111025755592436E-8\\ \hline
            5.342105263157896 & 5.809136739717792 & 5.809136599819555 & 1.3989823699489534E-7\\ \hline
            5.605263157894738 & 6.12627593608889 & 6.126276866933119 & 9.30844229252159E-7\\ \hline
            5.8684210526315805 & 6.932697833277216 & 6.932686958522396 & 1.0874754820200394E-5\\ \hline
        \end{tabular}
        \centering
        \begin{tabular}{|l|l|}
            \hline
            \begin{tabular}[c]{@{}l@{}}Среднее значение\\модуля разности\end{tabular} & \begin{tabular}[c]{@{}l@{}}Среднеквадратическое\\отклонение\end{tabular}  \\
            \hline
            1.6329960155097842E-6 & 2.2251060889537518E-11\\
            \hline
        \end{tabular}\end{table}
    \begin{figure}[h!]
        \centering
        \includegraphics[height=0.5\linewidth]{LagrangeInterpolation20Nodes.png}
    \end{figure}

    \newpage \section{Метод сплайнов}
    \listinginput[1]{1}{SplineInterpolationMethod.java}
    \newpage \subsection{Результаты работы программы}

    При {n = 5}
    \begin{table}[H]
        \centering
        \begin{tabular}{|l|l|l|l|}
            \hline
            x      & \begin{tabular}[c]{@{}l@{}}Значение\\ исходной\\ функции \end{tabular}& \begin{tabular}[c]{@{}l@{}}Значение \\ аппроксимирующей \\функции\end{tabular} & Модуль разности    \\
            \hline1.625 & 3.199969101926415 & 4.064798501162483 & 0.8648293992360676\\ \hline
            2.875 & 4.641302123952923 & 4.962963403045815 & 0.32166127909289166\\ \hline
            4.125 & 6.681748104899654 & 5.4826304977969995 & 1.1991176071026546\\ \hline
            5.375 & 5.815591599755577 & 7.358719249884802 & 1.5431276501292253\\ \hline
        \end{tabular}
        \centering
        \begin{tabular}{|l|l|}
            \hline
            \begin{tabular}[c]{@{}l@{}}Среднее значение\\модуля разности\end{tabular} & \begin{tabular}[c]{@{}l@{}}Среднеквадратическое\\отклонение\end{tabular}  \\
            \hline
            0.9821839838902098 & 1.1676304621269085\\
            \hline
        \end{tabular}\end{table}
    \begin{figure}[h!]
        \centering
        \includegraphics[height=0.7\linewidth]{SplineInterpolationMethod5Nodes.png}
    \end{figure}

\newpage    При {n = 10}

    \begin{table}[H]
        \centering
        \begin{tabular}{|l|l|l|l|}
            \hline
            x      & \begin{tabular}[c]{@{}l@{}}Значение\\ исходной\\ функции \end{tabular}& \begin{tabular}[c]{@{}l@{}}Значение \\ аппроксимирующей \\функции\end{tabular} & Модуль разности    \\
            \hline1.2777777777777777 & 2.0963095791094863 & 2.2441041068769025 & 0.1477945277674162\\ \hline
            1.8333333333333335 & 4.068842417577174 & 4.029511134531119 & 0.03933128304605482\\ \hline
            2.3888888888888893 & 5.351818613060926 & 5.324779306484609 & 0.02703930657631748\\ \hline
            2.9444444444444446 & 4.509608957056896 & 4.524005831299794 & 0.01439687424289815\\ \hline
            3.5 & 4.579715305532132 & 4.61199090371618 & 0.03227559818404746\\ \hline
            4.055555555555555 & 6.4768592675207755 & 6.454759726199684 & 0.02209954132109182\\ \hline
            4.611111111111111 & 7.021098663348462 & 6.991322669339731 & 0.02977599400873121\\ \hline
            5.166666666666666 & 5.926700217810059 & 5.96304216338731 & 0.036341945577250456\\ \hline
            5.722222222222221 & 6.441889396276311 & 6.432328186454178 & 0.00956120982213271\\ \hline
        \end{tabular}
        \centering
        \begin{tabular}{|l|l|}
            \hline
            \begin{tabular}[c]{@{}l@{}}Среднее значение\\модуля разности\end{tabular} & \begin{tabular}[c]{@{}l@{}}Среднеквадратическое\\отклонение\end{tabular}  \\
            \hline
            0.039846253393993364 & 0.0031286037641875286\\
            \hline
        \end{tabular}\end{table}

    \begin{figure}[h!]
        \centering
        \includegraphics[height=0.7\linewidth]{SplineInterpolationMethod10Nodes.png}
    \end{figure}
    ~\newline
    ~\newline

    \newpage При {n = 20}

    \begin{table}[H]
        \centering
        \begin{tabular}{|l|l|l|l|}
            \hline
            x      & \begin{tabular}[c]{@{}l@{}}Значение\\ исходной\\ функции \end{tabular}& \begin{tabular}[c]{@{}l@{}}Значение \\ аппроксимирующей \\функции\end{tabular} & Модуль разности    \\
            \hline1.131578947368421 & 1.8971563783952827 & 1.9186028984850418 & 0.021446520089759114\\ \hline
            1.394736842105263 & 2.380175273338317 & 2.3757815533735243 & 0.004393719964792542\\ \hline
            1.657894736842105 & 3.3352805820183207 & 3.3370503756812506 & 0.0017697936629299527\\ \hline
            1.9210526315789471 & 4.413265354985494 & 4.412287888838614 & 9.774661468799906E-4\\ \hline
            2.1842105263157894 & 5.1700065109846385 & 5.1689743947921905 & 0.0010321161924480293\\ \hline
            2.447368421052631 & 5.335182016483303 & 5.334143213401955 & 0.001038803081348405\\ \hline
            2.7105263157894735 & 4.975835530135844 & 4.9755576491712015 & 2.778809646422076E-4\\ \hline
            2.973684210526315 & 4.459148386662987 & 4.459756287793426 & 6.079011304391813E-4\\ \hline
            3.2368421052631575 & 4.236951533859102 & 4.2380971827204785 & 0.0011456488613763582\\ \hline
            3.499999999999999 & 4.57971530553213 & 4.580718637393869 & 0.0010033318617397313\\ \hline
            3.7631578947368416 & 5.417527415252814 & 5.417795816027524 & 2.6840077470957624E-4\\ \hline
            4.026315789473683 & 6.38222065592946 & 6.381596291418915 & 6.243645105454831E-4\\ \hline
            4.289473684210526 & 7.025786742458655 & 7.024635978923501 & 0.0011507635351533096\\ \hline
            4.552631578947368 & 7.085909405284797 & 7.08492528647481 & 9.841188099866116E-4\\ \hline
            4.815789473684211 & 6.6415640268940175 & 6.641283536032329 & 2.804908616882429E-4\\ \hline
            5.078947368421053 & 6.066655901825569 & 6.0674142128801245 & 7.583110545557048E-4\\ \hline
            5.342105263157896 & 5.809136739717792 & 5.809854113538882 & 7.173738210894953E-4\\ \hline
            5.605263157894738 & 6.12627593608889 & 6.128883891047753 & 0.0026079549588633455\\ \hline
            5.8684210526315805 & 6.932697833277216 & 6.926861634704293 & 0.005836198572922946\\ \hline
        \end{tabular}
        \centering
        \begin{tabular}{|l|l|}
            \hline
            \begin{tabular}[c]{@{}l@{}}Среднее значение\\модуля разности\end{tabular} & \begin{tabular}[c]{@{}l@{}}Среднеквадратическое\\отклонение\end{tabular}  \\
            \hline
            0.0024695346766247486 & 2.805481724152495E-5\\
            \hline
        \end{tabular}\end{table}
    \begin{figure}[h!]
        \centering
        \includegraphics[height=0.5\linewidth]{SplineInterpolationMethod20Nodes.png}
    \end{figure}

    \newpage\section{Метод наименьших квадратов}
    \listinginput[1]{1}{LeastSquaresMethod.java}
   \newpage \subsection{Результаты работы программы}
% \centering
    При {n = 5}\\
    \begin{table}[H]
        \centering
        \begin{tabular}{|l|l|l|l|}
            \hline
            x      & \begin{tabular}[c]{@{}l@{}}Значение\\ исходной\\ функции \end{tabular}& \begin{tabular}[c]{@{}l@{}}Значение \\ аппроксимирующей \\функции\end{tabular} & Модуль разности    \\
            \hline1.625 & 3.199969101926415 & 3.6033550295997765 & 0.40338592767336134\\ \hline
            2.875 & 4.641302123952923 & 5.215601843024135 & 0.5742997190712122\\ \hline
            4.125 & 6.681748104899654 & 5.8141289134331435 & 0.8676191914665106\\ \hline
            5.375 & 5.815591599755577 & 6.6606958513492565 & 0.8451042515936793\\ \hline
        \end{tabular}
        \centering
        \begin{tabular}{|l|l|}
            \hline
            \begin{tabular}[c]{@{}l@{}}Среднее значение\\модуля разности\end{tabular} & \begin{tabular}[c]{@{}l@{}}Среднеквадратическое\\отклонение\end{tabular}  \\
            \hline
            0.6726022724511909 & 0.4898761578582215\\
            \hline
        \end{tabular}\end{table}
    \begin{figure}[h!]
        \centering
        \includegraphics[height=0.7\linewidth]{LeastSquaresMethod5Nodes.png}
    \end{figure}
    \newpageПри {n = 10}\\

    \begin{table}[H]
        \centering
        \begin{tabular}{|l|l|l|l|}
            \hline
            x      & \begin{tabular}[c]{@{}l@{}}Значение\\ исходной\\ функции \end{tabular}& \begin{tabular}[c]{@{}l@{}}Значение \\ аппроксимирующей \\функции\end{tabular} & Модуль разности    \\
            \hline1.2777777777777777 & 2.0963095791094863 & 2.611204198101111 & 0.5148946189916246\\ \hline
            1.8333333333333335 & 4.068842417577174 & 3.761442328117119 & 0.3074000894600548\\ \hline
            2.3888888888888893 & 5.351818613060926 & 4.58751001644012 & 0.7643085966208059\\ \hline
            2.9444444444444446 & 4.509608957056896 & 5.163217204404984 & 0.653608247348088\\ \hline
            3.5 & 4.579715305532132 & 5.562373833346583 & 0.9826585278144506\\ \hline
            4.055555555555555 & 6.4768592675207755 & 5.858789844599787 & 0.6180694229209882\\ \hline
            4.611111111111111 & 7.021098663348462 & 6.126275179499466 & 0.8948234838489961\\ \hline
            5.166666666666666 & 5.926700217810059 & 6.438639779380486 & 0.5119395615704265\\ \hline
            5.722222222222221 & 6.441889396276311 & 6.86969358557773 & 0.42780418930141906\\ \hline
        \end{tabular}
        \centering
        \begin{tabular}{|l|l|}
            \hline
            \begin{tabular}[c]{@{}l@{}}Среднее значение\\модуля разности\end{tabular} & \begin{tabular}[c]{@{}l@{}}Среднеквадратическое\\отклонение\end{tabular}  \\
            \hline
            0.630611859764095 & 0.4404908728561391\\
            \hline
        \end{tabular}\end{table}
    \begin{figure}[h!]
        \centering
        \includegraphics[height=0.7\linewidth]{LeastSquaresMethod10Nodes.png}
    \end{figure}
   \newpage При {n = 20}\\

    \begin{table}[H]
        \centering
        \begin{tabular}{|l|l|l|l|}
            \hline
            x      & \begin{tabular}[c]{@{}l@{}}Значение\\ исходной\\ функции \end{tabular}& \begin{tabular}[c]{@{}l@{}}Значение \\ аппроксимирующей \\функции\end{tabular} & Модуль разности    \\
            \hline1.131578947368421 & 1.8971563783952827 & 2.17839360135191 & 0.2812372229566271\\ \hline
            1.394736842105263 & 2.380175273338317 & 2.8218615743956224 & 0.44168630105730555\\ \hline
            1.657894736842105 & 3.3352805820183207 & 3.383206532687023 & 0.0479259506687022\\ \hline
            1.9210526315789471 & 4.413265354985494 & 3.8693571608569637 & 0.5439081941285306\\ \hline
            2.1842105263157894 & 5.1700065109846385 & 4.287242143536301 & 0.8827643674483374\\ \hline
            2.447368421052631 & 5.335182016483303 & 4.643790165355884 & 0.6913918511274195\\ \hline
            2.7105263157894735 & 4.975835530135844 & 4.945929910946569 & 0.02990561918927437\\ \hline
            2.973684210526315 & 4.459148386662987 & 5.2005900649392105 & 0.7414416782762236\\ \hline
            3.2368421052631575 & 4.236951533859102 & 5.414699311964659 & 1.177747778105557\\ \hline
            3.499999999999999 & 4.57971530553213 & 5.595186336653769 & 1.0154710311216393\\ \hline
            3.7631578947368416 & 5.417527415252814 & 5.748979823637399 & 0.33145240838458445\\ \hline
            4.026315789473683 & 6.38222065592946 & 5.883008457546391 & 0.4992121983830691\\ \hline
            4.289473684210526 & 7.025786742458655 & 6.004200923011609 & 1.0215858194470453\\ \hline
            4.552631578947368 & 7.085909405284797 & 6.119485904663906 & 0.9664235006208912\\ \hline
            4.815789473684211 & 6.6415640268940175 & 6.2357920871341275 & 0.40577193975989\\ \hline
            5.078947368421053 & 6.066655901825569 & 6.360048155053134 & 0.293392253227565\\ \hline
            5.342105263157896 & 5.809136739717792 & 6.4991827930517765 & 0.690046053333984\\ \hline
            5.605263157894738 & 6.12627593608889 & 6.6601246857609055 & 0.5338487496720159\\ \hline
            5.8684210526315805 & 6.932697833277216 & 6.849802517811382 & 0.08289531546583451\\ \hline
        \end{tabular}
        \centering
        \begin{tabular}{|l|l|}
            \hline
            \begin{tabular}[c]{@{}l@{}}Среднее значение\\модуля разности\end{tabular} & \begin{tabular}[c]{@{}l@{}}Среднеквадратическое\\отклонение\end{tabular}  \\
            \hline
            0.562005696440763 & 0.4291555895780201\\
            \hline
        \end{tabular}\end{table}
    \begin{figure}[h!]
        \centering
        \includegraphics[height=0.5\linewidth] {LeastSquaresMethod20Nodes.png}
    \end{figure}
% \hline
\end{document}
