\documentclass[a4paper,12pt]{extarticle}
% !TEX encoding = IBM866
\usepackage[T2A]{fontenc}
\usepackage{listings}
\usepackage[dvipsnames]{xcolor}
\usepackage{algorithm2e}
\usepackage{array}
\usepackage{moreverb}
\usepackage{multirow}

\usepackage[russian]{babel}
\usepackage{amsthm,amsmath,amsfonts,amssymb}
\usepackage[final]{graphicx}
\usepackage{float}
\usepackage{geometry}
\geometry
{
a4paper,
total={210mm,297mm},
left=20mm,
right=20mm,
top=25mm,
bottom=20mm,
}

\usepackage{tikz}
\usepackage{pgfplots}
\usepackage[final]{showkeys}

\usepackage{caption}
\DeclareCaptionLabelSeparator{dot}{. }
\captionsetup{labelsep=dot}

\usepackage{hyperref}
\usepackage{lastpage}
\usepackage{fancyhdr}



\begin{document}


    \begin{titlepage}

        \begin{center}
            \centerline{\Large\rm Министерство науки и высшего образования}
            \centerline{\Large\rm Федеральное государственное бюджетное образовательное}
            \centerline{\Large\rm учреждение высшего образования}
            \centerline{\Large\rm <<Московский государственный технический университет}
            \centerline{\Large\rm имени~Н.~Э.~Баумана}
            \centerline{\Large\rm (национальный исследовательский университет)>>}
            \centerline{\Large\rm (МГТУ~им.~Н.~Э.~Баумана)}
            \hrulefill
        \end{center}

        \begin{figure}[h!]
            \centering
            \includegraphics[height=0.4\linewidth]{picture0}
        \end{figure}

        \begin{center}
            \centerline{\Large\rm Факультет <<Фундаментальные науки>>}
            \centerline{\Large\rm Кафедра ФН1 <<Высшая математика>>}
            \centerline{\Large\rm Дисциплина <<Методы оптимизации и вариационное исчисление>>}
        \end{center}

        \begin{center}
            \textsc{\textbf{\Huge Отчет}}\\
            \textsc{\textbf{\large по лабораторным работам №1--2}}\\
            \textsc{\textbf{\large Вариант 19}}\\
        \end{center}

        \vspace{3em}

        {
        \large
        \hbox to 17cm {Преподаватель \hspace{45pt} \hrulefill \hspace{75pt} Васильев~Н.~С.}
        \vspace{-7pt}
        \hbox{{\small\it \hspace{178pt} подпись, инициалы}}
        \hbox{}
        \hbox to 17cm {Студент группы ФН1--51Б \hrulefill \hspace{95pt} Терновой~Е.~А.}
        \vspace{-7pt}
        \hbox{{\small\it \hspace{178pt} подпись, инициалы}}
        }


        \vspace{\fill}

        \begin{center}
            \large	Москва \\2020
        \end{center}

    \end{titlepage}

    \setcounter{page}{2}
    \tableofcontents
    \vspace{\baselineskip}

    \newpage
    \section{Задание}

    \subsection{Лабораторная работа №1}
    Найти приближённое значение точки экстремума функции $f(x)$ с точностью с
    $\varepsilon = 0.001$ при помощи следующих численных методов оптимизации:

    \begin{enumerate}
        \item деление отрезка пополам
        \item золотое сечение
    \end{enumerate}

    \[
        f(x) = x^2 \cdot \log x
    \]

    Также провести сравнительный анализ скоростей сходимости методов.

    \subsection{Лабораторная работа №2}
    Найти приближённое значение глобальной точки экстремума функции $f(x)$
    с точностью $\varepsilon = 0.001$ при помощи следующих численных методов:

    \begin{enumerate}
        \item метод секущих
        \item метод касательных
    \end{enumerate}

    \[
        f(x) = x^2 \cdot \log x
    \]

    Провести сравнительный анализ скоростей сходимости методов. Представить таблично результаты первых $n=10$ итераций.

    \begin{figure}[h!]
        \centering
        \includegraphics[height=0.5\linewidth]{extrema.png}
        \caption{График исследуемой функции}
    \end{figure}

    \newpage
    \section{Код программы}

    \subsection{Деление отрезка пополам}
    \listinginput[1]{1}{SegmentDichotomyMethod.java}

    \newpage
    \subsection{Золотое сечение}
    \listinginput[1]{1}{GoldenRatioMethod.java}

    \newpage
    \subsection{Метод секущих}
    \listinginput[1]{1}{SecantMethod.java}

    \newpage
    \subsection{Метод касательных}
    \listinginput[1]{1}{TangentsMethod.java}

    \newpage
    \section{Результаты работы программы}

    \subsection{Поиск локального минимума функции на отрезке $[0.2; 0.8]$}

    \begin{center}
        \begin{tabular}{ |c|c|c|c|c| }
            \hline
            Метод & $\varepsilon$ & $f(x_{min})$ & $x_{min}(n)$ & $\Delta t$, с \\
            \hline
            \multirow{10}{9em}{\centerline{Дихотомия отрезка}} & \multirow{10}{5em}{\centerline{$0.001$}} 		  & \multirow{10}{10em}{\centerline{$-0.18393972058571076$}} &  		       $0.5$ & \multirow{10}{5em}{\centerline{$6.2\cdot10^{-4}$}} \\
            &  		  &  & $0.65$ &  \\
            &  		  &  & $0.575$ &  \\
            & 		  &  & $0.6125$ &  \\
            &  		  &  & $0.59375$ &  \\
            &  		  &  & $0.603125$ &  \\
            &  		  &  & $0.6078125$ &  \\
            & 		  &  & $0.60546875$ &  \\
            &  		  &  & $0.606640625$ &  \\
            &  		  &  & $0.6060546875$ &  \\
            \hline
        \end{tabular}
    \end{center}

    \begin{center}
        \begin{tabular}{ |c|c|c|c|c| }
            \hline
            Метод & $\varepsilon$ & $f(x_{min})$ & $x_{min}(n)$ & $\Delta t$, с \\
            \hline
            \multirow{10}{9em}{\centerline{Золотое сечение}} & \multirow{10}{5em}{\centerline{$0.001$}} 		  & \multirow{10}{10em}{\centerline{$-0.18393972058067223$}} &  		       $0.685$ & \multirow{10}{5em}{\centerline{$9.3\cdot10^{-4}$}} \\
            &  		  &  & $0.6145898033750316$ &  \\
            &  		  &  & $0.5437694101250946$ &  \\
            & 		  &  & $0.5875388202501892$ &  \\
            &  		  &  & $0.6145898033750316$ &  \\
            &  		  &  & $0.5978713763747792$ &  \\
            &  		  &  & $0.6082039324993691$ &  \\
            & 		  &  & $0.6018180616237067$ &  \\
            &  		  &  & $0.6057647468726342$ &  \\
            &  		  &  & $0.6082039324993691$ &  \\
            \hline
        \end{tabular}
    \end{center}


    \newpage
    \subsection{Поиск глобального минимума функции}



    \begin{center}
        \begin{tabular}{ |c|c|c|c|c| }
            \hline
            Метод & $\varepsilon$ & $f(x_{min})$ & $x_{min}(n)$ & $\Delta t$, с \\
            \hline
            \multirow{10}{9em}{\centerline{Секущие}} & \multirow{10}{5em}{\centerline{$0.001$}} 		  & \multirow{10}{10em}{\centerline{$-0.18393971733439896$}} &  		       $0.8$ & \multirow{10}{5em}{\centerline{$2.2\cdot 10^{-3}$}} \\
            &  		  &  & $0.42542535207764953$ &  \\
            & 		  &  & $0.7680207566862691$ &  \\
            &  		  &  & $0.47640742836957034$ &  \\
            &  		  &  & $0.7034657841103225$ &  \\
            &  		  &  & $0.5327556052144047$ &  \\
            & 		  &  & $0.6602655418137091$ &  \\
            &  		  &  & $0.5665278962134683$ &  \\
            &  		  &  & $0.6356424217273487$ &  \\
            &  		  &  & $0.5850549285073281$ &  \\
            \hline
        \end{tabular}
    \end{center}

    \begin{center}
        \begin{tabular}{ |c|c|c|c|c| }
            \hline
            Метод & $\varepsilon$ & $f(x_{min})$ & $x_{min}(n)$ & $\Delta t$, с \\
            \hline
            \multirow{4}{9em}{\centerline{Касательные}} & \multirow{4}{5em}{\centerline{$0.001$}} 		  & \multirow{4}{10em}{\centerline{$-0.18393972043398207$}} &  		       $0.6043034949999999$ & \multirow{4}{5em}{\centerline{$7.6\cdot 10^{-4}$}} \\
            &  		  &  & $0.6065919064285713$ &  \\
            & 		  &  & $0.6065183414285713$ &  \\
            &  		  &  & $0.6065183414285713$ & \\
            \hline
        \end{tabular}
    \end{center}

    \vspace{1in}
    \centering{\LaTeX}

\end{document}



